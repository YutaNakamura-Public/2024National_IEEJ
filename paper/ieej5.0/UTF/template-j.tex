%% v5.0 [2022/04/06]
%% 1. 和文「論文」(「解説」を含む)
%% 2. 和文「研究開発レター」
%% の各タイプ別のテンプレートです

%% 1. 和文「論文」用テンプレート
\documentclass[fleqn]{ieej}
\usepackage[defaultsups]{newtxtext}
\usepackage[varg]{newtxmath}
\usepackage[superscript,nomove]{cite}
\usepackage{graphicx}

\FIELD{}
\YEAR{2022}
\NO{1}
\jtitle{}
%\jtitle[]{}
\etitle{}
\authorlist{%
 \authorentry{}{}{}{}
% \authorentry{}{}{}{}
% \authorentry{}{}{}{}
}
\affiliate[]{ \\ }{ \\ }
%\affiliate[]{ \\ }{ \\ }
%\affiliate[]{ \\ }{ \\ }

%\received{}{}{}
%\revised{}{}{}
%\NoteOnArticle{本論文は……を加筆修正したものである。}
%\NoteIntConf{本論文は2020年 国際会議〇〇〇〇で発表した論文(文献(1) \copyright 2020 IEEJ)を加筆修正したものである。}

\begin{document}
\begin{abstract}

\end{abstract}
\begin{jkeyword}

\end{jkeyword}
\begin{ekeyword}

\end{ekeyword}
\maketitle

\section{}


\begin{thebibliography}{99}% 文献が10以上のとき99,10未満のとき9
\bibitem{}
\bibitem{}
\end{thebibliography}

\appendix
\section{}

%\acknowledgment % 謝辞

\begin{biography}
\profile{}{}{}
%\profile{}{}{}
%\profile*{}{}{}
\end{biography}
\end{document}


%% 2. 和文「研究開発レター」用テンプレート
\documentclass[letter,fleqn]{ieej}
\usepackage[defaultsups]{newtxtext}
\usepackage[varg]{newtxmath}
\usepackage[superscript,nomove]{cite}
\usepackage{graphicx}

\FIELD{}
\YEAR{2022}
\NO{1}
\jtitle{}
%\jtitle[]{}
\etitle{}
\authorlist{%
 \authorentry{}{}{}{}
% \authorentry{}{}{}{}
% \authorentry{}{}{}{}
}
\affiliate[]{ \\ }{ \\ }
%\affiliate[]{ \\ }{ \\ }
%\affiliate[]{ \\ }{ \\ }

%\received{}{}{}
%\revised{}{}{}

\begin{document}
\begin{abstract}

\end{abstract}
\begin{jkeyword}

\end{jkeyword}
\begin{ekeyword}

\end{ekeyword}
\maketitle

\section{}


\begin{thebibliography}{99}% 文献が10以上のとき99,10未満のとき9
\bibitem{}
\bibitem{}
\end{thebibliography}

\appendix
\section{}

%\acknowledgment % 謝辞

\begin{biography}
\profile{}{}{}
%\profile{}{}{}
%\profile*{}{}{}
\end{biography}
\end{document}

